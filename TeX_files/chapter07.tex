\chapter{Miscellaneous}
\maketitle

A place for all the random things you learn that you don't necessarily have a place for elsewhere in the notes, but you want to put them somewhere. Like the days when you read a lot of papers and you don't want to forget what you've learned.

\subsection*{3 October 2017}
One paper I read today was about using some flow equation to derive a new nuclear energy density functional (not - so far as I can tell - based on Skyrme or Gogny [at least not yet]). The nice thing about this scheme is that it has a built in truncation scheme, which is great for uncertainty quantification and error analysis and whatnot. I don't really understand what they did but it'll be interesting to see if it leads to anything better (although looking back on it, I don't see that they've specified a potential, so perhaps we'll be stuck with Skyrme or Gogny on some level after all...) See https://arxiv.org/pdf/1709.09143.pdf

I also spent a fair bit of time reading through a paper about correlated prompt fission data in transport simulations. What these codes try to do is predict the number, energy distribution, and maybe spin/angular momentum/directional/etc. properties of prompt neutrons, gamma rays, emitted in fission, given a fragment yield and a total kinetic energy as an input (basically). The mechanism on which these codes are based is Hauser-Feshbach, which is a reactions formalism that handles compound nuclei: given some total cross section, try to reconstruct the possible input channels and their weights (or something like that). There's a lot of stuff I don't understand about it, but at least it gives me something I can refer to when people ask me questions about scission neutrons and fragment angular distributions and whatnot. See https://arxiv.org/pdf/1710.00107.pdf

I showed Witek my cluster plots for 294Og and he wasn't at all surprised that the krypton protons took forever to coalesce into their shell configuration, but he did point out what I'd already basically noticed: that it's apparently easier for heavier nuclei to emit clusters.